
\newcommand\guy[1]{\og #1\fg}
\newcommand\kel{KelBillet}


\section{Sc\'{e}narios d'utilisation collect\'{e}s par le groupe C}
\label{groupeC:utilisation}
Nous rappelons que la cat\'{e}gorie d'utilisateurs cibl\'{e}e par le groupe C
concerne les 25-40 ans.


%************************
\subsection{Entretien avec un commercial de 40 ans}

\begin{description}
\item [Intervieweur :] Damien Le Guen
\item [Secr\'{e}taire :] Xavier Fraboulet
\item [Personne interrog\'{e}e, sa fonction et son \^{a}ge :] FH, commercial dans le domaine informatique, 43 ans
\item [Lieu :] Rennes
\item [Date et dur\'{e}e :] 02/11/2014 - 25 min
\item [Contexte :] Entretien sur un PC fixe dans un bureau au calme avec clavier/souris et \'{e}cran large, sous Windows XP et Firefox \'{e}quip\'{e} d'Adblock.
\end{description}


\begin{enumerate}
\item Apr\`{e}s \^{e}tre entr\'{e} sur le site via Google, nous demandons \`{a} l'utilisateur de faire une recherche pour Quimper-Paris pour le 6 novembre. Il s'ex\'{e}cute en entrant les noms de ville et en cliquant sur les suggestions d\`{e}s que propos\'{e}es. L'utilisateur veut aussi faire une recherche de train et clique en cons\'{e}quence sur la case \`{a} cocher \guy{Voyages-SNCF}, ce qui ouvre instantan\'{e}ment une fen\^{e}tre Pop-up vers la recherche sur Voyages-SNCF. L'utilisateur parcourt la liste d'horaires disponibles, au final sans passer par KelBillet.


\item A notre demande, l'utilisateur retourne sur la fen\^{e}tre de KelBillet (toujours sur la page principale) pour s\'{e}lectionner un autre billet. Il d\'{e}cide de chercher un covoiturage et clique sur l'ic\^{o}ne ronde \guy{Covoiturage} ouvrant la page de recherche sp\'{e}cifique pour Covoiturage, ce qui lui demande de retaper \`{a} nouveau la recherche. La personne clique ensuite sur \guy{Rechercher} pour obtenir la liste de r\'{e}sultats. L'utilisateur n'a ensuite pas de difficult\'{e} \`{a} choisir un covoiturage parmi les 3 disponibles en cliquant sur \guy{Plus d'infos}, ce qui ouvre un nouvel onglet vers BlaBlaCar.


\item A cet instant, la situation est donc la suivante : dans la m\^{e}me fen\^{e}tre, un onglet KelBillet.com et un second  BlaBlaCar (ce dernier \'{e}tant l'onglet courant actif), et dans une seconde fen\^{e}tre un onglet vers Voyages-SNCF. Nous demandons \`{a} la personne de revenir modifier sa recherche sur KelBillet. Il se retrouve alors perdu pendant une dizaine de secondes entre les deux fen\^{e}tres, en ne retrouvant plus le bon onglet vers KelBillet. 

L'utilisateur finit par retrouver le bon onglet et arrive sans difficult\'{e} \`{a} trouver \guy{Modifier la recherche}. Modification du trajet avec une arriv\'{e}e \`{a} Lyon. Apr\`{e}s rechargement de la page et d\'{e}but de scrolling vers le bas, l'utilisateur a un instant d'h\'{e}sitation lorsque la page se rafra\^{i}chit automatiquement de nouveau en remontant en haut de page. Il s\'{e}lectionne ensuite le trajet lui convenant. Cependant, il ne saisit pas la diff\'{e}rence entre les prix \guy{Covoiturage} et \guy{Voiture} affich\'{e}s en haut de page, il d\'{e}cide alors de scroller jusqu'en bas pour constater que Voiture correspond \`{a} \guy{Voiture personnelle} et que le prix affich\'{e} est donc une estimation de co\^{u}t.


\item Nous lui demandons de changer la date du d\'{e}part \`{a} quelques jours plus tard, il s'ex\'{e}cute en entrant cette fois au clavier les chiffres du jour sans passer par le calendrier affich\'{e}, puis lance la recherche. L'utilisateur ne remarque pas que la date n'a en r\'{e}alit\'{e} pas \'{e}t\'{e} modifi\'{e}e et correspond toujours \`{a} la recherche pr\'{e}c\'{e}dente, nous lui demandons de refaire la m\^{e}me manipulation et constatons que les entr\'{e}es faites au clavier se s\'{e}lectionnent bien dans le calendrier mais qu'au moment de cliquer sur Rechercher, la date revient \`{a} celle pr\'{e}c\'{e}dente.


\item De lui-m\^{e}me, l'utilisateur d\'{e}cide de faire une recherche qu'il a l'habitude de faire, partant de Nantes et allant jusqu'\`{a} Munich. Il est d\'{e}\c{c}u de constater qu'aucun trajet en train n'est disponible. Pour comparer sur plusieurs jours, il d\'{e}cide de modifier la date de recherche \`{a} plusieurs reprises. Nous lui indiquons alors qu'une case \guy{Date flexible} est disponible, l'utilisateur ne l'ayant toujours pas remarqu\'{e}e. Il a alors du mal \`{a} comprendre le fonctionnement de l'affichage des dates flexibles, n'affichant les r\'{e}sultats que des recherches pr\'{e}c\'{e}dentes. L'utilisateur se retrouve bloqu\'{e}. Nous l'aidons en lui conseillant de cliquer sur la loupe d'une journ\'{e}e souhait\'{e}e pour lancer la recherche, il s'ex\'{e}cute mais se retrouve perturb\'{e} par le d\'{e}calage des dates dans le calendrier (affichage de la derni\`{e}re date s\'{e}lectionn\'{e}e en gris fonc\'{e} et d\'{e}placement de chaque date d'un cran)


\item Apr\`{e}s \^{e}tre retourn\'{e} sur une page de r\'{e}sultats par d\'{e}faut, nous demandons \`{a} l'utilisateur d'acc\'{e}der au site partenaire SNCF. Au lieu de cliquer sur le lien existant dans les r\'{e}sultats, il se souvient de l'ouverture de la pop-up en d\'{e}but d'entretien et d\'{e}cide donc directement de passer par la case \`{a} cocher pr\'{e}sente dans le cadre de modification de la recherche vers la recherche sur Voyages-SNCF.

\end{enumerate}

\paragraph{Bilan de l'entretien avec la personne interview\'{e}e}


\begin{itemize}
  \item [\textbf{Points positifs}] :
      \begin{itemize}
        \item La recherche est simple;
        \item Le bandeau en haut de r\'{e}sultats donnant le prix le moins cher pour chaque moyen de transport est utile;
        \item L'affichage de l'estimation du co\^{u}t en voiture personnelle est int\'{e}ressant;
        \item L'acc\`{e}s direct \`{a} Voyages-SNCF est utile.
      \end{itemize}
      
  \item [\textbf{Faiblesses}] :
      \begin{itemize}
        \item La recherche est difficile pour les dates flexibles ;
    \item Il est peu pratique de modifier la date et la \guy{Date flexible} n'aide pas ;
        \item Il n'y a pas de recherche possible de trajets en train \`{a} l'\'{e}tranger.
      \end{itemize}
      
  \item [\textbf{Suggestions}] :
      \begin{itemize}
        \item Il faudrait arranger diff\'{e}remment la section \guy{Date flexible} (couleurs am\`{e}nent de la confusion; pourquoi aucune autre journ\'{e}e n'est affich\'{e}e sans cliquer sur la loupe ?).
      \end{itemize}
\end{itemize}


%************************
\subsection{Entretien avec une ing\'{e}nieure chimiste de 27 ans}

\begin{description}
\item [Intervieweur :] Xavier Fraboulet
\item [Secr\'{e}taire :] Damien Le Guen
\item [Personne interrog\'{e}e, sa fonction et son \^{a}ge :] NS, ing\'{e}nieure chimiste, 27 ans
\item [Lieu :] Rennes
\item [Date et dur\'{e}e :] 25/10/2014 - 30 min
\item [Contexte :] Cet entretien a \'{e}t\'{e} effectu\'{e} en utilisant le navigateur web Firefox avec Adblock activ\'{e} (le plugin \'{e}tant pr\'{e}sent sur l'ordinateur de NS). L'utilisatrice est une habitu\'{e}e des sites de voyage (SNCF, BlaBlaCar et Air France). Cependant, elle n'a jamais eu l'occasion d'utiliser le site \kel. L'entretien s'est d\'{e}roul\'{e} dans une pi\`{e}ce calme. 
\end{description}


\begin{enumerate}
\item \textit{Depuis la page d'accueil : cherchez un aller simple Rennes - Paris le 19/12/2014. Vous souhaitez vous d\'{e}placer en covoiturage.}

NS commence \`{a} remplir la ville de d\'{e}part puis clique sur Rennes dans la liste d\'{e}roulante. Idem pour la ville d'arriv\'{e}e.
CNS clique sur l'ic\^{o}ne calendrier dans le champ date et s\'{e}lectionne la date.
NS h\'{e}site \`{a} cliquer dans la partie \guy{Rechercher aussi sur} sur \guy{BlaBlaCar}. Finalement, elle clique sur \guy{Rechercher} car elle pense qu'elle aura plus de choix de covoiturage.

Dans la nouvelle fen\^{e}tre, NS clique sur l'onglet \guy{Covoiturage} mais cela ne marche pas.
Elle scrolle dans la page jusqu'\`{a} la cat\'{e}gorie covoiturage et clique sur \guy{+ d'infos} pour acc\'{e}der au billet.



\item \textit{Cherchez un trajet aller-retour Paris - Lyon en train n'importe quel jour de la semaine prochaine.} 

NS clique sur le bouton \guy{Pr\'{e}c\'{e}dent} de Firefox pour revenir \`{a} la page d'accueil.
Elle remplit les champs villes et choisit une date la semaine d'apr\`{e}s. Elle clique sur \guy{Rechercher}.
Elle scrolle dans la page jusqu'\`{a} la cat\'{e}gorie train.


\item \textit{Supposons que vous vous soyez tromp\'{e}e, pouvez-vous modifier votre recherche pour finalement aller \`{a} Marseille ?}

NS clique sur le bouton \guy{Pr\'{e}c\'{e}dent} de Firefox pour revenir \`{a} la page d'accueil.
Elle modifie le champ ville d'arriv\'{e}e en \guy{Marseille} et clique sur \guy{Rechercher}.


\item \textit{Supposons que vous souhaitez obtenir une vue globale des r\'{e}sultats sur le mois, que faites-vous ?}

Elle remonte en haut de la page et la parcourt des yeux. Apr\`{e}s 20 secondes, elle clique dans le panneau droit sur \guy{Tous les prix du mois}. Elle arrive sur la vue globale du mois.

\item \textit{Pouvez-vous afficher les r\'{e}sultats sous forme de timeline ?}

NS commence \`{a} chercher la fonctionnalit\'{e} au niveau du calendrier. Apr\`{e}s 15 secondes, elle remonte en haut de la page. Elle cherche des yeux l'ic\^{o}ne \guy{Timeline} et finit par le trouver. Elle semble h\'{e}sitante en cliquant dessus.


\item \textit{Revenez maintenant \`{a} la vue par d\'{e}faut d'affichage de r\'{e}sultats, pouvez-vous trouver un moyen d'acc\'{e}der aux offres sur des sites partenaires ?}

Elle clique sur le bouton \`{a} c\^{o}t\'{e} de \guy{Timeline} pour revenir \`{a} la vue par d\'{e}faut.
Elle va directement vers la section \guy{Recherche directe} et clique sur le partenaire \guy{BlaBlaCar}.
Elle nous indique qu'elle avait d\'{e}j\`{a} rep\'{e}r\'{e} cette section lors de ces recherches pour la vue \guy{Timeline}.


\end{enumerate}

\paragraph{Bilan de l'entretien avec la personne interview\'{e}e}

\begin{itemize}
  \item [\textbf{Points positifs}] :
      \begin{itemize}
      	\item le site se souvient de la derni\`{e}re recherche ;
        \item les billets sont s\'{e}par\'{e}s par cat\'{e}gorie ;
        \item il y a un choix important de billets et de transporteurs.
      \end{itemize}
      
  \item [\textbf{Faiblesses}] :
      \begin{itemize}
      	\item sur l'\'{e}cran d'accueil, la partie \guy{Rechercher sur} porte \`{a} confusion : si on va vite on peut penser que les \textit{checkboxes} sont des filtres ;
      	\item NS n'aurait pas su que la timeline existait si on ne lui avait pas dit de la chercher ;
        \item les ic\^{o}nes de changement de vue (\guy{Timeline}) ne sont pas suffisamment informatives ;
        \item NS ne comprend pas pourquoi dans la vue \guy{Calendrier} il n'y a pas de prix pour toutes les dates.
      \end{itemize}
      
  \item [\textbf{Suggestions}] :
      \begin{itemize}
      	\item Ajouter l'heure de d\'{e}part dans le formulaire de recherche.
      \end{itemize}
\end{itemize}




%************************
\subsection{Entretien avec un conducteur de ligne de production de 38 ans}

\begin{description}
\item [Intervieweur : ] Rapha\"{e}l Baron
\item [Secr\'{e}taire : ] Benoit Travers
\item [Personne interrog\'{e}e, sa fonction et son \^{a}ge : ] GL, conducteur de ligne de production, 38 ans
\item [Lieu : ] RENNES
\item [Date et dur\'{e}e : ] 29/10/2014 - 45 min
\item [Contexte : ] La pi\`{e}ce est calme et bien \'{e}clair\'{e}e. Seuls la personne interrog\'{e}e, l'intervieweur et le secr\'{e}taire sont pr\'{e}sents dans la pi\`{e}ce. Le test du site s'effectue sur l'ordinateur portable de l'intervieweur sous Mac OS X et Mozilla Firefox avec Adblock. 

\end{description}


\begin{enumerate}

\item \textit{Depuis la page d'accueil, recherche d'un aller simple Rennes - Paris le 19/12/2014 en covoiturage.}

GL commence par cliquer sur les fl\`{e}ches pr\'{e}sentes dans les champs \og{}Ville de d\'{e}part\fg{} et \og{}Ville d'arriv\'{e}e\fg{} avant de se rendre compte que rien ne se produit. 
Il tape \og{}rennes\fg{} pour le champ \og{}Ville de d\'{e}part\fg{} et \og{}paris\fg{} pour le champ \og{}Ville d'arriv\'{e}e\fg{} sans utiliser les propositions de ville qui lui sont sugg\'{e}r\'{e}es. 
Pour choisir la date, GL utilise le calendrier qui lui est mis \`{a} disposition et il finalise sa recherche en cliquant sur \og{}Blablacar\fg{}. 
Une nouvelle fen\^{e}tre s'ouvre et l'utilisateur est un peu perturb\'{e}. 
De retour sur le site gr\^{a}ce \`{a} l'intervention de l'intervieweur, GL lance la recherche en cliquant sur \og{}Rechercher\fg{}.
Le r\'{e}sultat de la recherche s'affiche. GL clique sur l'onglet \og{}Covoiturage\fg{} mais il ne se passe rien.
L'utilisateur descend dans la page pour trouver une alternative et trouve le champ covoiturage.
 
\item \textit{Cherchez un trajet aller-retour Paris - Lyon en train n'importe quel jour de la semaine prochaine.} 

GL retourne sur la page d'accueil en cliquant sur l'ic\^{o}ne \og{}Home\fg{}. 
Il remplace Rennes par Paris dans le champ \og{}Ville de d\'{e}part\fg{} et Paris par Lyon dans le champ \og{}Ville d'arriv\'{e}e\fg{}. 
Il clique sur \og{}Rechercher\fg{} et choisit le billet de train d'occasion.


\item \textit{Supposons que vous vous soyez tromp\'{e}, pouvez-vous modifier votre recherche pour finalement aller \`{a} Marseille ?}

GL clique \`{a} nouveau sur l'ic\^{o}ne \og{}Home\fg{}.
Il remplace Lyon par Marseille dans le champ \og{}Ville d'arriv\'{e}e\fg{}. 
Il clique sur \og{}Rechercher\fg{}


\item \textit{Supposons que vous souhaitez obtenir une vue globale des r\'{e}sultats sur le mois, que faites-vous ?}

GL veut cliquer sur \og{}Tous les prix du mois\fg{} mais il est g\^{e}n\'{e} par le bandeau \og{}Recherche directe\fg{} qui se d\'{e}roule.


\item \textit{Pouvez-vous afficher les r\'{e}sultats sous forme de timeline ?}

L'intervieweur explique la notion de timeline \`{a} GL.
GL ne trouve la timeline par lui-m\^{e}me.


\item \textit{Revenez maintenant \`{a} la vue par d\'{e}faut d'affichage de r\'{e}sultats, pouvez-vous trouver un moyen d'acc\'{e}der aux offres sur des sites partenaires ?}

GL retourne sur la vue par d\'{e}faut en cliquant sur \og{}Vue classique\fg{}. 
Il clique sur un partenaire associ\'{e} \`{a} un billet et non sur l'un des partenaires de la liste \og{}Recherche directe\fg{}.


\end{enumerate}

\paragraph{Bilan de l'entretien avec la personne interview\'{e}e}

\begin{itemize}
	\item [\textbf{Points positifs}] :
  		\begin{itemize}
  			\item la pr\'{e}sentation des champs de recherche sur la page d'accueil est claire et bien visible ;
			\item si on retourne sur la page d'accueil, les champs de recherche sont encore compl\'{e}t\'{e}s avec l'ancienne recherche ;
			\item le site KelBillet est tr\`{e}s r\'{e}actif.
		\end{itemize}
	\item [\textbf{Faiblesses}] :
		\begin{itemize}
			\item la timeline est difficile \`{a} trouver ;
			\item GL ne voit pas l'int\'{e}r\^{e}t de la vue timeline par rapport \`{a} la vue classique ;
			\item les onglets \og{}Covoiturage\fg{}, \og{}Train\fg{} et \og{}Avion\fg{} ne fonctionnent pas si un bloqueur de publicit\'{e} est activ\'{e}. 
		\end{itemize}
	\item [\textbf{Suggestions}] :
		\begin{itemize}
			\item l'ajout de l'heure de d\'{e}part dans le formulaire de d\'{e}part peut \^{e}tre interessant afin de diminuer le nombre de r\'{e}sultats dans la recherche.
		\end{itemize}
\end{itemize}




%************************
\subsection{Entretien avec un commercial de 32 ans}

\begin{description}
\item [Intervieweur :] Thomas Fran\c{c}ois
\item [Secr\'{e}taire :] Rapha\"{e}l Baron
\item [Personne interrog\'{e}e, sa fonction et son \^{a}ge :] JB, commercial dans le domaine automobile, 32 ans
\item [Lieu :] Rennes
\item [Date et dur\'{e}e :] 28/10/2014 - 30 min
\item [Contexte :] Cet entretien a \'{e}t\'{e} effectu\'{e} sur un appareil mobile (Iphone 5) en utilisant le navigateur web Safari. L'utilisateur est un utilisateur occasionnel de sites de voyage, notamment du site d'Air France. Il n'a jamais eu l'occasion d'utiliser le site \kel. L'entretien s'est d\'{e}roul\'{e} dans une pi\`{e}ce calme et lumineuse. 
\end{description}


\begin{enumerate}
\item \textit{Cherchez un trajet aller simple Rennes - Paris le 19/12/2014 en covoiturage.}

JB commence par \'{e}crire enti\`{e}rement \guy{rennes}, puis clique sur Rennes dans la liste d\'{e}roulante. Pour la ville d'arriv\'{e}e, JB s\'{e}lectionne Paris apr\`{e}s avoir seulement entr\'{e} \guy{par}.
JB s\'{e}lectionne la date gr\^{a}ce au calendrier, en s'\'{e}tonnant qu'il n'y ait pas \guy{de truc pour taper la date}.
JB h\'{e}site \`{a} cocher le bouton qui a pour l\'{e}gende \guy{BlaBlaCar}, puis d\'{e}cide de ne pas le faire.
Il lance finalement la recherche en cliquant sur \guy{Rechercher}.

Dans la nouvelle fen\^{e}tre, JB descend jusqu'\`{a} trouver les offres de covoiturage. Il s\'{e}lectionne le trajet le moins cher et clique sur \guy{+ d'infos} pour acc\'{e}der au billet.



\item \textit{Finalement, vous souhaitez partir le 22/12, que faites-vous ?} 

JB clique sur \guy{Modifier} afin de modifier sa recherche. Ensuite, il modifie la date gr\^{a}ce au calendrier, puis clique sur \guy{Rechercher}. Pour terminer, il descend dans la nouvelle page jusqu'aux offres de covoiturage, et choisit la moins ch\`{e}re.


\item \textit{Modifiez votre recherche pour aller maintenant \`{a} Marseille, vous souhaitez \'{e}galement obtenir
un trajet retour le 23/12 dans la m\^{e}me recherche.}

JB clique sur le bouton \guy{Modifier}, puis modifie les villes de d\'{e}part et d'arriv\'{e}e. Ensuite, il coche l'option \guy{Retour}, puis modifie la date via le calendrier. JB clique ensuite sur \guy{Lancer la recherche}, puis descend jusqu'aux offres de covoiturage dans la nouvelle fen\^{e}tre et s\'{e}lectionne la moins ch\`{e}re.


\item \textit{Pouvez-vous filtrer uniquement les billets d'avion sur la page des r\'{e}sultats ?}

JB remonte en haut de la page, puis clique sur l'ic\^{o}ne repr\'{e}sentant un avion. Lorsque les r\'{e}sultats sont filtr\'{e}s, il dit \guy{Ah oui, \c{c}a filtre}.

\item \textit{Pouvez-vous afficher vos recherches pr\'{e}c\'{e}dentes ?}

JB commence \`{a} chercher la fonctionnalit\'{e} sur la page o\`{u} il est. Apr\`{e}s pr\`{e}s d'une minute sans r\'{e}sultat, il retourne sur l'accueil, mais ne trouve toujours pas l'option. Au final, il abandonne et dit \guy{Il faut peut-\^{e}tre \^{e}tre connect\'{e} pour \c{c}a ?}.


\end{enumerate}

\paragraph{Bilan de l'entretien avec la personne interview\'{e}e}

\begin{itemize}
  \item [\textbf{Points positifs}] :
      \begin{itemize}
      	\item le bandeau de recherche est bien visible ;
        \item l'interface est claire et l'utilisation de la recherche intuitive ;
        \item la pr\'{e}visualisation des diff\'{e}rents moyens de transport disponibles lors de l'auto compl\'{e}tion est int\'{e}ressante ;
        \item la publicit\'{e} est discr\`{e}te et ne g\^{e}ne pas la navigation.
      \end{itemize}
      
  \item [\textbf{Faiblesses}] :
      \begin{itemize}
      	\item sur l'\'{e}cran d'accueil, la partie \guy{Rechercher sur} embrouille l'utilisateur ;
      	\item l'historique des recherches n'est pas facilement accessible ;
        \item il est p\'{e}nible de devoir s\'{e}lectionner une ville que l'on a d\'{e}j\`{a} enti\`{e}rement orthographi\'{e}e ;
        \item les filtres ne sont pas assez visibles.
      \end{itemize}
      
  \item [\textbf{Suggestions}] :
      \begin{itemize}
      	\item Afficher la pr\'{e}visualisation des prix pour les jours pr\'{e}c\'{e}dents la recherche, et pas seulement pour ceux d'apr\`{e}s.
      \end{itemize}
\end{itemize}




%************************
\subsection{Entretien avec une assistante \`{a} la personne de 27 ans}

\begin{description}
\item [Intervieweur :] Benoit Travers
\item [Secr\'{e}taire :] Thomas Fran\c{c}ois
\item [Personne interrog\'{e}e, sa fonction et son \^{a}ge :] SL, assistante \`{a} la personne, 27 ans
\item [Lieu :] Rennes
\item [Date et dur\'{e}e :] 30/10/2014 - 30 min
\item [Contexte :] Cet entretien a \'{e}t\'{e} effectu\'{e} sur l'ordinateur portable (Windows 8) de la personne interrog\'{e}e en utilisant le navigateur web Chrome, avec le plugin AdBlock activ\'{e}. L'utilisatrice consulte des sites de voyage de mani\`{e}re occasionnelle, notamment le site de la SNCF. Elle n'a jamais eu l'occasion d'utiliser le site \kel. L'entretien s'est d\'{e}roul\'{e} dans une pi\`{e}ce calme et lumineuse. 
\end{description}


\begin{enumerate}
\item \textit{Depuis la page d'accueil : cherchez un aller simple Rennes - Paris le 19/12/2014. Vous souhaitez vous d\'{e}placer en covoiturage.}

SL commence \`{a} taper \guy{rennes} dans le champ \og{}Ville de d\'{e}part\fg{}, puis clique sur Rennes dans la liste d\'{e}roulante. Pour la ville d'arriv\'{e}e, SL s\'{e}lectionne Paris apr\`{e}s avoir seulement entr\'{e} \guy{par}.
SL s\'{e}lectionne la date gr\^{a}ce au calendrier.
SL clique sur le bouton qui a pour l\'{e}gende \guy{BlaBlaCar}, ce qui provoque une redirection vers le site partenaire. SL est perdue et ne sait plus quoi faire. L'intervieweur intervient pour la faire revenir sur le site \kel.
Elle lance alors la recherche en cliquant sur \guy{Rechercher}.

Dans la nouvelle fen\^{e}tre, SL clique sur l'onglet covoiturage, rien ne se passe. Elle r\'{e}essaye une nouvelle fois sans plus de succ\`{e}s. SL fait alors d\'{e}filer la page jusqu'aux offres de covoiturage. Elle indique un trajet, sans tenter de cliquer dessus.



\item \textit{Cherchez un trajet aller-retour Paris - Lyon en train n'importe quel jour de la semaine prochaine.}

SL retourne sur la page d'accueil en cliquant sur l'ic\^{o}ne \og{}Home\fg{}. 
Elle remplace Rennes par Paris dans le champ \og{}Ville de d\'{e}part\fg{} et Paris par Lyon dans le champ \og{}Ville d'arriv\'{e}e\fg{}, en tapant le d\'{e}but des noms puis en les s\'{e}lectionnant dans la liste d\'{e}roulante. 
Elle clique sur \og{}Rechercher\fg{}.

Elle ne trouve pas d'indication que les billets sont aller-retour. SL revient en arri\`{e}re (touche \og{pr\'{e}c\'{e}dent} du navigateur) puis s\'{e}lectionne l'onglet \og{}Aller-retour\fg{}. Elle choisit la date du 15 pour le retour puis clique sur \og{}Rechercher\fg{}.

SL clique sur l'onglet \og{}Bus\fg{} parce qu'il contient le prix le plus bas et rien ne se passe. SL d\'{e}file alors jusqu'aux offres de bus.


\item \textit{Supposons que vous vous soyez tromp\'{e}e, pouvez-vous modifier votre recherche pour finalement aller \`{a} Marseille ?}

SL clique sur le bouton pr\'{e}c\'{e}dent du navigateur. 
Elle remplace Lyon par Marseille dans le champ \og{}Ville d'arriv\'{e}e\fg{}. 
Elle clique sur \og{}Rechercher\fg{}.
Elle s'\'{e}tonne de voir des trajets Nantes-Marseille.


\item \textit{Supposons que vous souhaitez obtenir une vue globale des r\'{e}sultats sur le mois, que faites-vous ?}

SL clique sur le bouton \og{}Modifier la recherche\fg{} puis sur l'onglet \og{}Aller simple\fg{}. Elle ouvre et referme le calendrier. Elle clique sur \og{}dates flexibles\fg{} puis sur \og{}Rechercher\fg{} et dit \og{}Oula\fg{}.


\item \textit{Pouvez-vous afficher les r\'{e}sultats sous forme de timeline ?}

SL clique sur le bouton indiqu\'{e} Beta (timeline) par hasard avant que l'intervieweur ne lui ai demand\'{e} puis clique sur le bouton adjacent pour revenir \`{a} la vue classique.


\item \textit{Revenez maintenant \`{a} la vue par d\'{e}faut d'affichage de r\'{e}sultats, pouvez-vous trouver un moyen d'acc\'{e}der aux offres sur des sites partenaires ?}

SL clique sur le logo du site puis sur le bouton \og{}Rechercher\fg{}.
Elle d\'{e}file l'\'{e}cran enti\`{e}rement puis annonce ne pas avoir trouv\'{e} les partenaires.


\end{enumerate}

\paragraph{Bilan de l'entretien avec la personne interview\'{e}e}

\begin{itemize}
  \item [\textbf{Points positifs}] :
      \begin{itemize}
        \item les villes d\'{e}part et arriv\'{e}e sont faciles \`{a} renseigner ;
        \item revenir \`{a} l'accueil est simple ;
        \item les offres ont une bonne visibilit\'{e} et sont bien s\'{e}par\'{e}es par type de v\'{e}hicule.
      \end{itemize}
      
  \item [\textbf{Faiblesses}] :
      \begin{itemize}
        \item l'option aller-retour est peu discernable ;
        \item l'option dates variables n'est pas claire;
        \item il est impossible de cliquer sur les onglets pour acc\'{e}der aux offres apr\`{e}s une recherche ;
        \item la redirection vers les sites partenaires est subite ;
        \item le bandeau \og{}Bons plans\fg{} est \'{e}nervant.
      \end{itemize}
      
  \item [\textbf{Suggestions}] :
      \begin{itemize}
      	\item Pouvoir indiquer l'heure de d\'{e}part et/ou d'arriv\'{e}e serait un plus.
      \end{itemize}
\end{itemize}




%************************
\subsection{Entretien avec un professeur des \'{e}coles de 28 ans}

\begin{description}
\item [Intervieweur :] Nicolas Charpentier
\item [Secr\'{e}taire :] Benoit Travers
\item [Personne interrog\'{e}e, sa fonction et son \^{a}ge :] BG, Professeur des \'{e}coles, 28 ans
\item [Lieu :] Foug\`{e}res
\item [Date et dur\'{e}e :] 11/11/2014 - 24 min
\item [Contexte :] Entretien sur un t\'{e}l\'{e}phone portable Android, navigateur chrome pour mobile, dans le salon de l'interview\'{e}, sur une table. L'utilisateur a d\'{e}j\`{a} utilis\'{e} son mobile pour r\'{e}server des covoiturages sur BlaBlaCar. Nous lui avons indiqu\'{e} le site KelBillet, il se retrouve sur la fen\^{e}tre d'accueil du site pour mobile.
\end{description}


\begin{enumerate}
\item \textit{Cherchez un trajet aller simple Rennes - Paris le 19/12/2014 en covoiturage.}

BG clique sur la case \guy{Ville de d\'{e}part} et renseigne les trois premi\`{e}res lettres \guy{ren}, et clique sur \guy{Rennes} qui est directement propos\'{e} par le site. BG r\'{e}it\`{e}re son op\'{e}ration pour la ville d'arriv\'{e}e, il renseigne \guy{par} et clique sur \guy{Paris}.
BG teste le bouton \guy{Aller / Retour}, voit qu'une date de retour est propos\'{e}e au premier appui, il r\'{e}appuie sur ce bouton pour le faire dispara\^{i}tre. BG renseigne alors la \guy{date aller}, fait d\'{e}filer la liste d\'{e}roulante pour le vendredi 19 d\'{e}cembre et lance la recherche en appuyant sur le bouton \guy{Lancer la recherche}.

Il inspecte la fen\^{e}tre r\'{e}sultat de haut en bas, retourne en haut et clique sur le logo repr\'{e}sentant une voiture, il r\'{e}inspecte le r\'{e}sultat et clique sur le covoiturage le moins cher \guy{Rennes - Evry}. Il se rend compte sur le site de BlaBlaCar qu'il s'agit de Rennes - Evry. Il fait un retour en arri\`{e}re et clique sur le deuxi\`{e}me covoiturage \guy{Rennes - Paris}.


\item \textit{Finalement, vous souhaitez partir le 22/12, que faites-vous ?} 

BG fait un retour arri\`{e}re pour retourner sur le site Kelbillet, il \guy{slide} sur le calendrier jusqu'au \guy{Lun 22 D\'{e}c.}, s\'{e}lectionne le logo repr\'{e}sentant la voiture et clique sur le premier r\'{e}sultat.


\item \textit{Modifiez votre recherche pour aller maintenant \`{a} Marseille, vous souhaitez \'{e}galement obtenir un trajet retour le 23/12 dans la m\^{e}me recherche.}

BG fait un retour arri\`{e}re, se rend compte qu'il ne retourne en arri\`{e}re que pour la date. Il clique alors sur \guy{Modifier}, clique sur \guy{Paris}, remplit {mar} et clique sur {Marseille}. Il appuie ensuite sur le bouton \guy{aller / retour}, fait d\'{e}filer la liste d\'{e}roulante, clique sur le \guy{Mardi 23 D\'{e}cembre} et lance la recherche.


\item \textit{Pouvez-vous filtrer uniquement les billets d'avion sur la page des r\'{e}sultats ?}

BG s\'{e}lectionne directement le logo repr\'{e}sentant un avion.

\item \textit{Pouvez-vous afficher vos recherches pr\'{e}c\'{e}dentes ?}

BG descend jusqu'en bas de la fen\^{e}tre r\'{e}sultat et nous indique ses derni\`{e}res recherches {Rennes - Paris le 19/12/2014} et {Rennes - Paris le 22/12/2014}.


\end{enumerate}

\paragraph{Bilan de l'entretien avec la personne interview\'{e}e}

\begin{itemize}
  \item [\textbf{Points positifs}] :
      \begin{itemize}
      	\item l'interface est intuitive ;
        \item apr\`{e}s une recherche, on se sent \`{a} l'aise avec le site ;
        \item le filtre pour s\'{e}lectionner son trajet est utile.
      \end{itemize}
      
  \item [\textbf{Faiblesses}] :
      \begin{itemize}
      	\item la liste d\'{e}roulante pour les dates est peu intuitive.
      \end{itemize}
      
  \item [\textbf{Suggestions}] :
      \begin{itemize}
      	\item Il faudrait am\'{e}liorer la pr\'{e}sentation des dates.
      \end{itemize}
\end{itemize}
