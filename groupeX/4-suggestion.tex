
\section{Suggestions d'utilisateurs collect\'{e}es par le groupe C}

Nous rappelons que la cat\'{e}gorie d'utilisateurs cibl\'{e}e par le groupe C concerne les 25-40 ans. Lors de nos entretiens, nous avons r\'{e}colt\'{e} les suggestions suivantes :

\begin{enumerate}

\item \textbf{Mettre le bouton vers la timeline plus en \'{e}vidence}, un certain nombre d'utilisateurs ayant trouv\'{e} des difficult\'{e}s \`{a} y acc\'{e}der.
\item Lors du clic sur une des ic\^{o}nes rondes\footnote{4 ic\^{o}nes rondes sous le formulaire de \og{rechercher} : train, covoiturage, avion et bus.} sur la page d'accueil, cela serait plus intuitif de \textbf{conserver la recherche d\'{e}j\`{a} entr\'{e}e}. En effet ces liens sont parfois confondus avec des boutons pour lancer la recherche uniquement pour un certain type de transport.
\item \textbf{Coh\'{e}rence lors de la redirection}, si la recherche n\'{e}cessite de rediriger l'utilisateur vers un site externe, il serait int\'{e}ressant que dans chaque cas cela se fasse vers soit une nouvelle fen\^{e}tre, soit un nouvel onglet. Mais non les deux, car cela embrouille les utilisateurs peu habitu\'{e}s \`{a} la gestion des fen\^{e}tres de navigateur.
\item \textbf{Inclure le choix de l'heure dans le champ de recherche} ou proposer \`{a} l'utilisateur d'affiner ult\'{e}rieurement sa recherche par heure de d\'{e}part.
\item \textbf{Remplacer les listes d\'{e}roulantes par des calendriers} sur le site mobile, car cela n'est pas parfois pas pratique pour trouver une date lointaine.
\item \textbf{Permettre la s\'{e}lection de la date au clavier} dans le champ de recherche principal.
\item \textbf{Pouvoir obtenir les trajets vers l'\'{e}tranger en train}, car cela est pratique pour pouvoir comparer sur des grands trajets avec l'avion.

\end{enumerate}